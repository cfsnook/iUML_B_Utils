\section{Class Attributes}
\label{sec:classdiagrams-attributes}

Classes may have attributes to model properties that the instances of that class exhibit.
Attributes are equivalent to associations except that the target set is usually a simple set that is not appropriate to draw as a class.
Also attributes default to total functions (\code{Total = true, Functional = true}) as it is usual for all instances to have a single value for an attribute.

The \code{Type} of an attribute is given by a string in the properties field.
The type may a built in type such as \code{INT}, \code{NAT}, \code{BOOL} or be an expression that identifies a set of values in the model (e.g. name of a data item that is a set).
 
The 

\paragraph{Initial Value } 
The initial value field is a Rodin Keyboard text field. 
The string in the text field is used as the intialisation value for the owning class attribute (or association) but can be interreted in several ways:
\begin{itemize}
	\item If the attribute is owned by a variable instance class, this field is currently ignored. (It should be used in a constructor).
	\item If the attribute elaborates a constant, this field is ignored. 
	\item If the attribute elaborates a variable, an action is added to the Initialisation event to intialise the attribute. as follows:
	\begin{itemize}
		\item if the text field begins with one of the Event-B assignment operators it is used to assign to the complete attribute relation (i.e. expression = attribute.name+initialValue.String)
		\item otherwise the text is assumed to contain the value to be assigned to each and every instance \\(i.e. expression = attribute.name+":="+class.name+"X"+initialValue.String)
	\end{itemize}	
\end{itemize}

%%% Local Variables:
%%% mode: latex
%%% TeX-master: "iuml-b-classdiagrams-user_manual"
%%% End:
