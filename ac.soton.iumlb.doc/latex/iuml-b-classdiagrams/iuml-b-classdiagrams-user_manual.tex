\documentclass[a4paper,10pt]{article}
%\documentclass[preview=false]{standalone}
% %%%% Define new conditional \ifplastex
% %%%% This ensure that the file complied normally without plasTeX.
\usepackage{plastex}

\ifplastex\else
\usepackage{standalone} %%%% Only load standalone when not in plasTeX
\standalonetrue
\fi

% % Package for typesetting URLs
\usepackage{url}

\usepackage{import}

\usepackage{iumlb}

\usepackage{multirow}

% Package for including figures
\usepackage{graphicx}

% Package for sub-floats (e.g. figures)
\usepackage{subfig}

% % Title
\title{iUML-B Class Diagrams User Manual} 

% % Author
\author{Colin Snook\\University of Southampton}

% % Date
\date{%
	Version 0.0.1\\%
	\today%
}

\begin{document}
	\ifplastex%
	\maketitle% Make title if in plasTeX mode.
	\else%
	\ifstandalone%
	\maketitle % Make title if in standalone mode.
	\else%
	\fi%
	\fi%
	
iUML-B Class diagrams are used to visualise and model data in an entity relationship style.
Class Diagrams provide a way to model data as sets (classes) of objects.
Elements of the class diagram (class, attribute, association) \emph{elaborate} (i.e. link to) a data item (Carrier Set, Constant, Variable) in the Event-B.
The class diagram translator generates properties such as axioms or invariants in the relevant context or machine.
Linking to an existing data element, rather than create a new one, allows more flexibility.
The elaborated data item must be in-scope via sees or extends as shown in Figs.~\ref{fig:CDContainment1}, \ref{fig:CDContainment2} and 	\ref{fig:CDContainment3}.
	
	\begin{figure}[!htbp]
		\centering
		\ifplastex
		\includegraphics[width=500]{figures/containment_1.png}
		\else
		\includegraphics[width=.5\textwidth]{figures/containment_1.png}
		\fi
		\caption{Class Diagram contained in a base level Context}
		\label{fig:CDContainment1}
	\end{figure}
	
	
	\begin{figure}[!htbp]
		\centering
		\ifplastex
		\includegraphics[width=500]{figures/containment_2.png}
		\else
		\includegraphics[width=.5\textwidth]{figures/containment_2.png}
		\fi
		\caption{Class Diagram contained in an extended Context}
		\label{fig:CDContainment2}
	\end{figure}
	
	
	\begin{figure}[!htbp]
		\centering
		\ifplastex
		\includegraphics[width=500]{figures/containment_3.png}
		\else
		\includegraphics[width=.5\textwidth]{figures/containment_3.png}
		\fi
		\caption{Class Diagram contained in a Machine}
		\label{fig:CDContainment3}
	\end{figure}

%% subsections for each element type in a class diagram
	
	\section{Classes}
\label{sec:classdiagrams-classes}

This section describes the main properties of classes. 
These properties are shown and can be edited in the \code{Overview} tab of the properties view which is shown when a class is selected on the class diagram canvas.
The other properties tabs of the class relate to relationships between classes or children of classes and are described in later sections.

Classes represent sets of elements. 
The set can be a given type (\code{Carrier Set}), a constant subset (\code{Constant}) or a variable subset (\code{Variable}).
The kind of the class is decided by linking it to an existing data element in the Event-B model.
This link is called \code{elaborates} and is set in the properties view when the class is selected in the diagram.
Fig.~\ref{fig:ClassCreation} shows a class that has just been created using the palette. 
It is selected so that its properties are shown in the property view.
It has no name and does not elaborate any data element.
The diagram renders the un-elaborated class with a green background and an icon representing a class.

\emph{Tip: If you leave a class without any elaboration and just type a string in the name field, the string will be used verbatim in the translation. 
	This can be useful for constructing types such as cross products in associations and attributes.}

\begin{figure}[!htbp]
	\centering
	\ifplastex
	\includegraphics[width=1000]{figures/ClassCreation.png}
	\else
	\includegraphics[width=1\textwidth]{figures/ClassCreation.png}
	\fi
	\caption{Class just created but not elaborated}
	\label{fig:ClassCreation}
\end{figure}

To set the elaboration of a class, use the buttons in the property view.
\begin{itemize}
	\item \command{Link Data} - adds a link to an existing data element in a machine or context
	\item \command{Un-link Data} - removes a link to a data element but leaves the data element
	\item \command{Create \& Link} - creates a new data element in a machine or context and links to it
	\item \command{Un-link \& Delete} - removes a link to a data element and also deletes the data element
\end{itemize}
\emph{Note: the \command{Create \& Link} and \command{Un-link \& Delete} buttons provide shortcuts to save you editing the machine or context. 
	The data elements are created/deleted from the machine/context as soon as you save the diagram. 
	These changes are NOT part of the translation.}

After setting the elaboration, the \code{Name} of the class, and its \code{Data Kind} property (Carrier Set, Constant or Variable) are updated automatically from the name and kind of the elaborated data. 
The rendering of the class changes automatically in the diagram to a yellow background to indicate that the elaborates property is set.
The icon in front of the class name changes to match that of the Event-B Explorer for the kind of data item that is elaborated (purple star = Carrier Set, yellow disc = Constant, green disc = Variable).
Fig.~\ref{fig:ClassElaboration} shows a class that \code{Elaborates} a carrier set.

\begin{figure}[!htbp]
	\centering
	\ifplastex
	\includegraphics[width=1000]{figures/ClassElaboration.png}
	\else
	\includegraphics[width=1\textwidth]{figures/ClassElaboration.png}
	\fi
	\caption{Class that elaborates a Carrier Set}
	\label{fig:ClassElaboration}
\end{figure}

The \code{Self Name} property reserves an identifier for the contextual instance of the class (similar to this or self in O-O programming languages).
This identifier is used and/or recognised as the bound variable in quantifications over the class instances.
It is also used as the name of an automatically generated event parameter when the class has methods.
The default \code{Self Name}, shown in Fig.~\ref{fig:ClassElaboration}, is constructed by prepending \code{this\_} to the class name.


\begin{figure}[!htbp]
	\centering
	\ifplastex
	\includegraphics[width=1000]{figures/ClassEnumeration.png}
	\else
	\includegraphics[width=1\textwidth]{figures/ClassEnumeration.png}
	\fi
	\caption{Class with instances that for an enumeration}
	\label{fig:ClassEnumeration}
\end{figure}


The \code{Instances} property can be used to specify the instances of the class.
For example, Fig.~\ref{fig:ClassEnumeration} shows the class instances specified as a set of labelled items.
The set brackets are recognised by the translation which generates a data element for each label and an axiom to ensure their uniqueness and coverage of the class.
The generated Event-B is:

\CONTEXT{Enumerations}{}{}
\SETS{
	\Set{QUARKS}{}
}
\CONSTANTS{
	\Constant{up}{instance of class QUARKS}
	\Constant{down}{instance of class QUARKS}
	\Constant{charm}{instance of class QUARKS}
	\Constant{strange}{instance of class QUARKS}
	\Constant{top}{instance of class QUARKS}
	\Constant{bottom}{instance of class QUARKS}
}
\AXIOMS{
	\Axiom{instancesOf\_QUARKS}{false}{$partition(QUARKS, \{up\},\{down\},\{charm\},\{strange\},\{top\},\{bottom\})$}{}
}
\END

The \code{refines} property allows you to select a class of a refined class diagram (i.e., in a machine refined by this diagram's machine).
Refinement of class diagrams will be explained in a later section.

 

	%
	\section{Class Attributes}
\label{sec:classdiagrams-attributes}

Classes may have attributes to model properties that the instances of that class exhibit.
Attributes are equivalent to associations except that the target set is usually a simple set that is not appropriate to draw as a class.
Also attributes default to total functions (\code{Total = true, Functional = true}) as it is usual for all instances to have a single value for an attribute.

The \code{Type} of an attribute is given by a string in the properties field.
The type may a built in type such as \code{INT}, \code{NAT}, \code{BOOL} or be an expression that identifies a set of values in the model (e.g. name of a data item that is a set).
 
The 

\paragraph{Initial Value } 
The initial value field is a Rodin Keyboard text field. 
The string in the text field is used as the intialisation value for the owning class attribute (or association) but can be interreted in several ways:
\begin{itemize}
	\item If the attribute is owned by a variable instance class, this field is currently ignored. (It should be used in a constructor).
	\item If the attribute elaborates a constant, this field is ignored. 
	\item If the attribute elaborates a variable, an action is added to the Initialisation event to intialise the attribute. as follows:
	\begin{itemize}
		\item if the text field begins with one of the Event-B assignment operators it is used to assign to the complete attribute relation (i.e. expression = attribute.name+initialValue.String)
		\item otherwise the text is assumed to contain the value to be assigned to each and every instance \\(i.e. expression = attribute.name+":="+class.name+"X"+initialValue.String)
	\end{itemize}	
\end{itemize}

%%% Local Variables:
%%% mode: latex
%%% TeX-master: "iuml-b-classdiagrams-user_manual"
%%% End:

	%
	\section{Associations}
\label{sec:classdiagrams-associations}
	%
	\section{Class Methods}
\label{sec:classdiagrams-methods}
	%
	\section{Class Constraints}
\label{sec:classdiagrams-constraints}
	%
	\section{Class Statemachines}
\label{sec:classdiagrams-statemachines}

\begin{itemize}
	\item statemachine containment
	
	\item what it gives you.


\end{itemize}
	
\end{document}

%%% Local Variables:
%%% mode: latex
%%% TeX-master: t
%%% End: